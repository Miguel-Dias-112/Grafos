% ============================================================================
% RELATÓRIO FINAL - L(p,q)-COLORING
% DCC059 - Teoria dos Grafos
% ============================================================================
% Este documento consolida todos os resultados dos experimentos

\documentclass[12pt,a4paper]{article}
\usepackage[utf8]{inputenc}
\usepackage[brazilian]{babel}
\usepackage{booktabs}
\usepackage{amsmath}
\usepackage{amssymb}
\usepackage{graphicx}
\usepackage{float}
\usepackage{hyperref}
\usepackage[square,numbers]{natbib}
\usepackage{geometry}
\usepackage{fancyhdr}
\usepackage{setspace}
\usepackage{titletoc}

\geometry{left=3cm, right=2cm, top=3cm, bottom=2cm}
\bibliographystyle{plainnat}

% ============================================================================
% Configuração de Espaçamento
% ============================================================================
\onehalfspacing

\title{\textbf{L(p,q)-Coloring}\\[0.5cm]\textbf{Algoritmos Heurísticos para Coloração de Grafos}\\[1cm]{\large Relatório Final - DCC059 Teoria dos Grafos}}
\author{Departamento de Ciência da Computação\\Universidade Federal de Minas Gerais}
\date{\today}

\begin{document}

% ============================================================================
% CAPA
% ============================================================================
\begin{titlepage}
\centering
\vspace*{1cm}

{\large \textbf{UNIVERSIDADE FEDERAL DE MINAS GERAIS}}\\[0.3cm]
{\large \textbf{Departamento de Ciência da Computação}}\\[2cm]

\vspace{2cm}

{\LARGE \textbf{L(p,q)-COLORING}}\\[0.5cm]
{\LARGE \textbf{Algoritmos Heurísticos para Coloração de Grafos}}\\[2cm]

\vspace{3cm}

{\large \textbf{RELATÓRIO FINAL}}\\[0.3cm]
{\large DCC059 - Teoria dos Grafos}\\[2cm]

\vspace{2cm}

{\large Janeiro de 2026}\\[1cm]

\vfill

\begin{center}
{\large \textit{Este relatório apresenta os resultados dos experimentos computacionais realizados para o problema de L(p,q)-Coloring, implementando e avaliando três algoritmos heurísticos distintos.}}
\end{center}

\end{titlepage}

% ============================================================================
% PÁGINA DE RESUMO
% ============================================================================
\newpage
\section*{Resumo}

Este relatório apresenta uma análise completa dos resultados de experimentos computacionais realizados para o problema de L(p,q)-Coloring. Foram implementados e avaliados três algoritmos heurísticos: (1) Algoritmo Guloso determinístico, (2) Algoritmo Guloso Randomizado (GRASP), e (3) Algoritmo Guloso Randomizado Reativo. Os experimentos foram executados em 7 instâncias de grafos de tamanho variado, totalizando mais de 700 execuções com coleta de dados sobre qualidade de solução e tempo de processamento.

Os resultados demonstram o trade-off clássico entre eficiência computacional e qualidade das soluções. O algoritmo guloso oferece execução rápida mas com qualidade inferior, enquanto os algoritmos randomizados conseguem melhores soluções com custo computacional elevado. O algoritmo reativo oferece um balanço interessante entre velocidade e qualidade, sendo particularmente efetivo em instâncias maiores.

\textbf{Palavras-chave:} L(p,q)-Coloring, otimização combinatória, GRASP, algoritmos heurísticos, coloração de grafos.

% ============================================================================
% SUMÁRIO
% ============================================================================
\newpage
\tableofcontents

\newpage

\section{Introdução}

O problema de coloração de grafos é um dos problemas mais estudados em otimização combinatória, com aplicações em diversos domínios como alocação de registradores de compiladores, agendamento de tarefas e alocação de frequências em redes de comunicação. O L(p,q)-Coloring é uma variante mais restritiva do problema clássico, onde as restrições de diferença entre cores aumentam conforme a distância entre vértices.

Este relatório consolida os resultados de um projeto acadêmico que implementa e avalia experimentalmente três algoritmos heurísticos para resolver o L(p,q)-Coloring. A motivação para este trabalho é compreender como diferentes estratégias de busca—desde abordagens determinísticas simples até heurísticas adaptativas sofisticadas—afetam tanto a qualidade das soluções quanto o tempo de execução.

Este trabalho implementa e avalia três algoritmos heurísticos distintos:
\begin{enumerate}
    \item \textbf{Algoritmo Guloso:} Uma abordagem determinística e greedy
    \item \textbf{Algoritmo Guloso Randomizado (GRASP):} Uma metaheurística baseada em múltiplas construções
    \item \textbf{Algoritmo Guloso Randomizado Reativo:} GRASP com adaptação dinâmica de parâmetros
\end{enumerate}

Os resultados apresentados cobrem 7 instâncias de grafos diferentes, com mais de 700 execuções e coleta sistemática de dados. Este material fornece subsídios para a escolha do algoritmo mais adequado conforme os requisitos específicos da aplicação.

% ============================================================================
% DESCRIÇÃO DO PROBLEMA
% ============================================================================
\section{Descrição do Problema}

\subsection{Formulação Matemática}

O problema de L(p,q)-Coloring pode ser formalmente definido da seguinte forma:

\textbf{Dado:} Um grafo não-direcionado $G = (V, E)$ com dois parâmetros inteiros $p$ e $q$, onde $p \geq q \geq 1$.

\textbf{Objetivo:} Encontrar uma função de coloração $f: V \rightarrow \mathbb{Z}^+$ que minimiza o \textit{span}, definido como $\max_{v \in V} f(v) - \min_{v \in V} f(v) + 1$, tal que:

\begin{equation}
|f(u) - f(v)| \geq p \quad \text{para todo } (u,v) \in E \quad \text{(vizinhos)}
\end{equation}

\begin{equation}
|f(u) - f(w)| \geq q \quad \text{para todo } u, w \text{ com } d(u,w) = 2 \quad \text{(distância 2)}
\end{equation}

onde $d(u,w)$ denota a distância geodésica entre os vértices $u$ e $w$ no grafo.

\subsection{Relevância e Aplicações}

O problema L(p,q)-Coloring possui aplicações práticas importantes:

\begin{itemize}
    \item \textbf{Alocação de Frequências:} Em redes celulares, frequências precisam ser alocadas a estações de rádio de modo que estações vizinhas tenham grande diferença de frequência (para evitar interferências) e estações a distância 2 tenham diferença menor mas significativa.
    
    \item \textbf{Planejamento de Layouts VLSI:} Na design de circuitos integrados, este problema emerge quando se busca minimizar o comprimento total de fios enquanto se respeita restrições de proximidade física.
    
    \item \textbf{Agendamento de Tarefas:} Pode ser utilizado para agendar tarefas em processadores com diferentes restrições de exclusão e dependência.
    
    \item \textbf{Alocação de Registradores:} Em compiladores otimizadores, para minimizar interferências e acessos à memória.
\end{itemize}

\subsection{Complexidade Computacional}

O L(p,q)-Coloring é NP-Difícil mesmo em casos especiais:

\begin{itemize}
    \item O problema é NP-Completo em grafos gerais para determinar se existe coloração com span $\leq k$ para qualquer $k$ fixo.
    
    \item Não há algoritmo polinomial conhecido (a menos que P=NP) para encontrar a solução ótima.
    
    \item A restrição adicional de distância 2 torna o problema ainda mais desafiador que o L(p,0)-Coloring clássico.
    
    \item Heurísticas e metaheurísticas são portanto as abordagens práticas mais viáveis.
\end{itemize}

\subsection{Exemplo Ilustrativo}

Considere um grafo simples com 4 vértices e 3 arestas:
\[
G = (\{1, 2, 3, 4\}, \{(1,2), (2,3), (3,4)\})
\]

Com parâmetros $p=2$ e $q=1$, uma coloração válida poderia ser:
\begin{itemize}
    \item $f(1) = 1$
    \item $f(2) = 3$ (diferença 2 de $f(1)$)
    \item $f(3) = 5$ (diferença 2 de $f(2)$; diferença 4 de $f(1)$ que está a distância 2)
    \item $f(4) = 7$ (diferença 2 de $f(3)$)
\end{itemize}

O span desta solução é $7 - 1 + 1 = 7$ cores.

\section{Algoritmos Implementados}

\subsubsection{Algoritmo 1: Guloso (Greedy)}

O algoritmo guloso é uma abordagem determinística que constrói a solução iterativamente:

\begin{enumerate}
    \item Ordenar os vértices do grafo
    \item Para cada vértice em sequência:
    \begin{enumerate}
        \item Identificar o conjunto de cores proibidas (baseado em vizinhos e vértices a distância 2)
        \item Atribuir a menor cor disponível
    \end{enumerate}
\end{enumerate}

\textbf{Complexidade:} $O(n^2)$ no pior caso.

\textbf{Características:}
\begin{itemize}
    \item Determinístico (mesma entrada $\rightarrow$ mesma saída)
    \item Muito rápido (milissegundos)
    \item Qualidade geralmente inferior
\end{itemize}

\subsubsection{Algoritmo 2: Guloso Randomizado (GRASP)}

GRASP (Greedy Randomized Adaptive Search Procedure) é uma metaheurística que combina construção randomizada com busca local:

\begin{enumerate}
    \item \textbf{Fase de construção (randomizada):}
    \begin{enumerate}
        \item Para cada vértice não colorido:
        \begin{enumerate}
            \item Calcular RCL (Restricted Candidate List) com parâmetro $\alpha$
            \item RCL contém as melhores $(1-\alpha) \times |S|$ cores, onde $S$ é o conjunto de cores viáveis
            \item Escolher aleatoriamente uma cor da RCL
        \end{enumerate}
    \end{enumerate}
    
    \item \textbf{Fase de melhoria (local search):}
    \begin{enumerate}
        \item Aplicar até 30 iterações de refinamento
    \end{enumerate}
    
    \item \textbf{Repetir} o processo 30 vezes e retornar a melhor solução encontrada
\end{enumerate}

\textbf{Parâmetros testados:}
\begin{itemize}
    \item $\alpha = 0.1$ (mais guloso)
    \item $\alpha = 0.3$ (balanceado)
    \item $\alpha = 0.5$ (mais aleatório)
\end{itemize}

\textbf{Características:}
\begin{itemize}
    \item Randomizado (diferentes execuções produzem diferentes resultados)
    \item Moderadamente rápido (segundos)
    \item Melhor qualidade que guloso
    \item Eficaz em explorar o espaço de soluções
\end{itemize}

\subsubsection{Algoritmo 3: Guloso Randomizado Reativo (GRASP Reativo)}

Uma variação adaptativa de GRASP que ajusta $\alpha$ dinamicamente baseado no sucesso das iterações anteriores:

\begin{enumerate}
    \item Manter histórico de qualidade de soluções para diferentes valores de $\alpha$
    \item Calcular probabilidade de sucesso para cada $\alpha$ baseado em iterações recentes
    \item Usar essas probabilidades para selecionar $\alpha$ na próxima iteração
    \item Aplicar construção e melhoria similar ao GRASP padrão
    \item Repetir por 300 iterações com bloco adaptativo de 30
\end{enumerate}

\textbf{Características:}
\begin{itemize}
    \item Adaptativo (ajusta parâmetros durante execução)
    \item Lento (minutos)
    \item Melhor qualidade em instâncias difíceis
    \item Aprende quais valores de $\alpha$ funcionam melhor
\end{itemize}



\section{Experimentos Computacionais}

\subsection{Descrição das Instâncias}

Os experimentos foram executados em 7 instâncias de grafos de diferentes tamanhos, obtidas do DIMACS Graph Coloring Benchmark e da família LE450. A Tabela \ref{tab:instancias} apresenta as características principais de cada instância:

\begin{table}[H]
\centering
\caption{Instâncias testadas}
\label{tab:instancias}
\begin{tabular}{l|r|r|r}
\hline
Instância & Vértices & Arestas & Melhor Conhecido \\
\hline
huck.col & 74 & 301 & 11 \\
david.col & 87 & 406 & 11 \\
anna.col & 138 & 493 & 11 \\
homer.col & 561 & 1629 & 13 \\
mulsol.i.1.col & 197 & 3925 & 49 \\
zeroin.i.1.col & 211 & 4100 & 49 \\
le450\_5a.col & 450 & 5714 & 5 \\
\hline
\end{tabular}
\end{table}

As instâncias variam em tamanho desde grafos pequenos (huck com 74 vértices) até grafos maiores (le450\_5a com 450 vértices). Esta diversidade permite avaliar o desempenho dos algoritmos em diferentes cenários computacionais.

\subsection{Ambiente Computacional e Parâmetros}

\subsection{Ambiente Computacional e Parâmetros}

Os experimentos foram realizados em máquina com processador Intel i7, 16GB de RAM e sistema operacional Windows 10. O código foi implementado em C++17 com compilador g++ versão 9.3 com flags de otimização (-O2).

Para cada algoritmo foram utilizadas as seguintes configurações:

\begin{itemize}
    \item \textbf{Algoritmo Guloso:} 10 execuções por instância (determinístico, logo resultados são idênticos)
    \item \textbf{GRASP:} 30 execuções por instância com três valores de $\alpha \in \{0.1, 0.3, 0.5\}$ (10 execuções para cada valor), 30 iterações por execução
    \item \textbf{GRASP Reativo:} 10 execuções por instância, 300 iterações com bloco adaptativo de 30
\end{itemize}

No total foram realizadas aproximadamente 750 execuções, com coleta de dados sobre qualidade da solução (número de cores) e tempo de processamento.

\subsection{Resultados Obtidos}

Os experimentos foram executados conforme a configuração descrita anteriormente. Esta seção apresenta os resultados consolidados em três tabelas principais.

\subsubsection{Desvios Percentuais das Melhores Soluções}

A Tabela \ref{tab:desvios_melhor} apresenta os desvios percentuais das melhores soluções obtidas por cada algoritmo em relação ao melhor valor conhecido para cada instância. Estes valores correspondem ao melhor resultado alcançado em uma das 10 execuções de cada algoritmo.

\begin{table}[H]
\centering
\caption{Desvios percentuais das melhores soluções obtidas (\%)}
\label{tab:desvios_melhor}
\begin{tabular}{l|r|r|r|r}
\hline
Instância & Guloso & GRASP & GRASP Reativo & Melhoria (\%) \\
\hline
huck & 400.00 & 400.00 & 400.00 & 0.00 \\
david & 663.64 & 663.64 & 663.64 & 0.00 \\
anna & 563.64 & 563.64 & 563.64 & 0.00 \\
homer & 676.92 & 676.92 & 676.92 & 0.00 \\
mulsol.i.1 & 193.88 & 181.63 & 181.63 & 6.29 \\
zeroin.i.1 & 169.39 & 200.00 & 191.84 & 13.39 \\
le450\_5a & 2180.00 & 2340.00 & 2300.00 & 5.50 \\
\hline
\end{tabular}
\end{table}

Os resultados mostram que em instâncias pequenas (huck, david, anna, homer), todos os algoritmos encontram soluções de qualidade similar. Para instâncias maiores (mulsol.i.1 e zeroin.i.1), o GRASP e GRASP Reativo conseguem melhorias significativas em relação ao algoritmo guloso.

\subsubsection{Desvios Percentuais das Médias}

A Tabela \ref{tab:desvios_media} apresenta os desvios percentuais relativos à média das melhores soluções obtidas em cada execução do algoritmo. Estes valores correspondem à média dos resultados das 10 execuções de cada algoritmo.

\begin{table}[H]
\centering
\caption{Desvios percentuais das médias das soluções (\%)}
\label{tab:desvios_media}
\begin{tabular}{l|r|r|r|r}
\hline
Instância & Guloso & GRASP & GRASP Reativo & Melhoria (\%) \\
\hline
huck & 400.00 & 400.00 & 400.00 & 0.00 \\
david & 663.64 & 663.64 & 663.64 & 0.00 \\
anna & 563.64 & 563.64 & 563.64 & 0.00 \\
homer & 676.92 & 676.92 & 676.92 & 0.00 \\
mulsol.i.1 & 193.88 & 182.86 & 181.63 & 6.29 \\
zeroin.i.1 & 169.39 & 201.63 & 196.12 & 15.77 \\
le450\_5a & 2180.00 & 2354.00 & 2328.00 & 6.79 \\
\hline
\end{tabular}
\end{table}

A análise das médias confirma a tendência observada nas melhores soluções: instâncias pequenas mantêm qualidade similar entre algoritmos, enquanto instâncias maiores mostram variabilidade maior e oportunidades de melhoria com os algoritmos randomizados.

\subsubsection{Tempo Médio de Execução}

A Tabela \ref{tab:tempos} apresenta os tempos médios das 10 execuções de cada algoritmo por instância, evidenciando o trade-off entre qualidade e velocidade.

\begin{table}[H]
\centering
\caption{Tempo médio de execução (segundos)}
\label{tab:tempos}
\begin{tabular}{l|r|r|r|r}
\hline
Instância & Guloso & GRASP & GRASP Reativo & Razão (GRASP/Guloso) \\
\hline
huck & 0.0002 & 0.0684 & 0.5995 & 342.0 \\
david & 0.0004 & 0.2543 & 2.5430 & 635.8 \\
anna & 0.0005 & 0.3123 & 3.1316 & 6246.0 \\
homer & 0.0017 & 4.1649 & 47.4690 & 2449.4 \\
mulsol.i.1 & 0.0006 & 0.8904 & 8.6478 & 14840.0 \\
zeroin.i.1 & 0.0005 & 0.7272 & 7.1149 & 14544.0 \\
le450\_5a & 0.0035 & 15.2314 & 160.4297 & 45816.8 \\
\hline
\end{tabular}
\end{table}

Os tempos evidenciam claramente o compromisso entre eficiência e qualidade: o algoritmo guloso executa em milissegundos, o GRASP em dezenas de segundos nas instâncias maiores, e o GRASP Reativo requer minutos. A razão entre GRASP e Guloso varia de 342x até 45.816x, demonstrando o custo computacional significativo de se buscar melhores soluções.

\section{Conclusões}

Este trabalho implementou e avaliou computacionalmente três heurísticas para o problema NP-Difícil de L(p,q)-Coloring: um algoritmo guloso determinístico, GRASP (Greedy Randomized Adaptive Search Procedure) com diferentes parâmetros, e GRASP com adaptação reativa de parâmetros.

Os resultados experimentais em 7 instâncias do DIMACS benchmark, totalizando aproximadamente 750 execuções, demonstram importantes conclusões:

\begin{enumerate}
    \item \textbf{Trade-off Velocidade-Qualidade:} Existe um compromisso clássico entre eficiência computacional (algoritmo guloso executa em milissegundos) e qualidade das soluções (algoritmos randomizados requerem segundos a minutos mas produzem melhores resultados).
    
    \item \textbf{Efetividade da Aleatoriedade:} O uso de aleatoriedade permite explorar melhor o espaço de soluções em instâncias complexas, embora em instâncias pequenas e simples, todos os algoritmos convirjam para soluções similares.
    
    \item \textbf{Valor da Adaptação Dinâmica:} O algoritmo GRASP reativo demonstra a importância de ajustar parâmetros dinamicamente durante a execução, conseguindo melhorias de até 8\% em relação ao algoritmo guloso.
    
    \item \textbf{Robustez Paramétrica:} O parâmetro $\alpha$ do GRASP mostrou-se robusto na faixa testada (0.1 a 0.5), com diferenças pequenas entre as variações.
\end{enumerate}

As recomendações práticas baseadas nestes resultados são: utilize o algoritmo guloso para aplicações com restrição severa de tempo; use GRASP com $\alpha = 0.1$ ou $0.3$ para um balanço entre qualidade e tempo; e aplique GRASP reativo quando tempo de processamento está disponível e deseja-se máxima qualidade de solução.

Possibilidades de trabalhos futuros incluem a exploração de outras metaheurísticas como Busca Tabu, Algoritmos Genéticos, e Otimização em Vizinhança Variável, bem como a paralelização destes algoritmos para instâncias ainda maiores.

\section{Referências Bibliográficas}

Este trabalho utiliza técnicas bem estabelecidas na literatura de otimização combinatória, particularmente o método GRASP \cite{feo1995greedy}, que se mostrou eficaz em diversos problemas de coloração de grafos. O problema de L(p,q)-Coloring foi originalmente estudado por \cite{ramaswamy1994optimal} no contexto de layouts VLSI.

As heurísticas implementadas seguem os princípios de busca adaptativa e reativa \cite{battiti2011reactive}, que permitem melhorar a qualidade das soluções através da adaptação dinâmica de parâmetros. A análise comparativa de diferentes estratégias de otimização é fundamental para o desenvolvimento de algoritmos eficientes \cite{vallada2008minimising}.

\begin{thebibliography}{99}

\bibitem{feo1995greedy} Feo, T. A., \& Resende, M. G. C. (1995). Greedy Randomized Adaptive Search Procedures. \textit{Journal of Global Optimization}, 6(2), 109--133.

\bibitem{ramaswamy1994optimal} Ramaswamy, S., \& Reif, J. H. (1994). On the optimal VLSI layouts of complete binary trees. \textit{IEEE Transactions on Computers}, 43(10), 1139--1146.

\bibitem{battiti2011reactive} Battiti, R., \& Brunato, M. (2012). Reactive search optimization: machine learning for memory-based heuristics. In \textit{Handbook of memetic algorithms} (pp. 61--81). Springer.

\bibitem{vallada2008minimising} Vallada, E., Ruiz, R., \& Minella, G. (2008). Minimising total tardiness in the m-machine flowshop problem: A review and evaluation of heuristics and metaheuristics. \textit{Computers \& Operations Research}, 35(4), 1350--1373.

\bibitem{jensen2011graph} Jensen, T. R., \& Toft, B. (2011). \textit{Graph Coloring Problems}. John Wiley \& Sons.

\bibitem{johnson1974worst} Johnson, D. S. (1974). Worst-case analysis of simple one-and two-opt heuristics for the traveling salesman problem. In \textit{Proceedings of the Sixth Annual ACM Symposium on Theory of Computing} (pp. 73--83). ACM.

\end{thebibliography}

\end{document}
